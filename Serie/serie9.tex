d-infk
Prof. Dr. Özlem Imamoglu
Analysis I
Serie 9
ETH Zürich
FS 2022
Nur die Aufgaben mit einem * werden korrigiert.
9.1. MC Fragen.
(a) Sei f(x) = cos 
1
x

. Wählen Sie die richtige Antwort.
□ lim x→+∞
f(x) = 0
□ lim x→+∞
f(x) = 1
□ lim x→+∞
f(x) = +∞
□ lim x→+∞
f(x) existiert nicht
(b) Sei D ⊂ R, f : D → R und nehmen Sie an, dass es m ∈ R und r : D → R eine
stetige Funktion gibt, so dass r(x0) = 0 und
f(x) = f(x0) + m(x − x0) + r(x)(x − x0),
wobei x0 ∈ D ein Häufungspunkt von D ist. Ist f in x0 differenzierbar?
□ Ja
□ Nein
□ Nicht genügend Informationen, um festzustellen.
(c) Definiere für x > 0
f(x) := lim inf
k→∞

min{x, x−1
}
k
.
Dann
□ f ist stetig und differenzierbar
□ f ist differenzierbar, aber nicht stetig
□ f ist stetig, aber nicht differenzierbar
□ f ist nicht stetig und nicht differenzierbar
1/2
ETH Zürich
FS 2022
Analysis I
Serie 9
d-infk
Prof. Dr. Özlem Imamoglu
*9.2. Link- und Rechtseitige Grenzwert. Bestimmen Sie die Link- und die
Rechtseitige Grenzwerte von
f(x) = sign(x) · cos2
(x)
in x = 0. Existiert limx→0
f(x)? Falls ja, bestimmen Sie diese Wert. Falls nein, erklären
Sie warum. Ist f eine stetige Funktion?
*9.3. Ableitung I. Berechnen Sie die Ableitungen der folgenden Funktionen.
(a) f : (0, π/2) → R, f(x) = ln(tan(x)).
(b) f : (0,∞) → R, f(x) := x
x
a
, wobei a ∈ R eine feste Zahl ist.
Hinweis: Es könnte hilfreich sein, x
x
a
als x
x
a
= e
ln x
x
a
zu schreiben.
9.4. Ableitung II.
(a) Sei f : R → R eine Funktion, die in x0 ∈ R differenzierbar ist. Sei n ∈ N, n ≥ 2.
Berechne
lim
h→0
f(x0 + nh) − f(x0 + (n − 2)h)
h
.
(b) Eine Funktion f : R → R heisst gerade (resp. ungerade), falls f(−x) = f(x)
(resp. f(−x) = −f(x)) gilt für alle x ∈ R.
Zeige: falls f auf ganz R differenzierbar ist, dass gilt:
(i) f gerade =⇒ f
′ ungerade.
(ii) f ungerade =⇒ f
′ gerade.
2/2
