d-infk
Prof. Dr. Özlem Imamoglu
Analysis I
Serie 8
ETH Zürich
FS 2022
Nur die Aufgaben mit einem * werden korrigiert.
8.1. MC Fragen.
(a) Sei D ⊂ R und fn : D → R eine Folge stetigen Funktionen. Wählen Sie die
richtige Aussagen.
□ Falls fn nach f gleichmässig konvergiert, konvergiert fn nach f punktweise.
□ Falls |fn(x)| < cn für cn ∈ R und jedes x ∈ D, dann konvergiert die Reihe
P∞
n=0 fn(x) punkteweise.
□ Falls |fn(x)| < cn für jedes x ∈ D und P∞
n=1 cn konvergiert, dann konvergiert die Reihe P∞
n=0 fn(x) punkteweise.
(b) Sei D ⊂ R eine Telmenge. Der Punkt x0 ∈ R ist ein Häufungspunkt von D, falls
□ x0 ∈ D;
□ für jedes δ > 0 gilt es ((x0 − δ, x0 + δ) \ {x0}) ∩ D ̸= ∅;
□ für jedes δ > 0 gilt es (x0 − δ, x0 + δ) ∩ D ̸= ∅.
(c) Wir betrachten die Funktionenfolge (fn) mit
fn : R≥0 → R, x 7→ (x
1/2 + n
−1
)
2
.
Welche der Aussagen gilt?
□ limn→∞ fn(x) = x für alle x ∈ R≥0
□ Die Funktionenfolge konvergiert gleichmässig.
□ Für alle M > 0 gilt, dass die Funktionenfolge fn|[0,M]
: [0, M] → R
gleichmässig konvergiert.
*8.2. Konvergenz von Funktionenfolgen. Konvergieren die folgenden Funktionenfolgen auf dem Intervall [0, 1] punktweise gegen eine Grenzfunktion f? Falls ja,
bestimmen Sie f und untersuche, ob die Konvergenz gleichmässig ist.
(a) fn(x) :=

1 +
x
n
2
, n ∈ N;
(b) fn(x) :=
sin x
n
;
(c) fn(x) := 1 + x
n
(1 − x)
n
, n ∈ N;
1/2
ETH Zürich
FS 2022
Analysis I
Serie 8
d-infk
Prof. Dr. Özlem Imamoglu
*8.3. Gleichmässigkonvergenz. fn : D → R sei beschränkt für jedes n ∈ N und fn
konvergiere gleichmässig gegen f auf D. Dann zeigen Sie, dass f : D → R beschränkt
ist.
8.4. Trigonometrische Funktion.
(a) Zeige, dass ∀x, y ∈ R
sin x − sin y = 2 sin 
x − y
2

cos 
x + y
2

(1)
cos x − cos y = −2 sin 
x − y
2

sin 
x + y
2

(2)
(b) Zeige dass sin: h
−
π
2
,
π
2
i
→ [−1, 1] eine strong monoton stetige bijektive Abbildung ist.
(c) Zeige für alle k ∈ N und 0 ≤ x ≤
q
(4k + 5)(4k + 6) gilt:
cos x ≥ 1 −
x
2
2! +
x
4
4! −
x
6
6! + ... −
x
2(2k+1)
[2(2k + 1)]!
2/2
