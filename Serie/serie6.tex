d-infk
Prof. Dr. Özlem Imamoglu
Analysis I
Serie 6
ETH Zürich
FS 2022
Nur die Aufgaben mit einem * werden korrigiert.
6.1. MC Fragen: Wählen Sie die richtigen Antworten.
(a) Kreuze die richtigen Aussagen an.
Seien f, g : D → R monoton wachsende Funktionen, D ⊆ R.
□ f · g : D → R ist monoton wachsend.
□ Angenommen g(x) ̸= 0 für alle x ∈ D. Dann ist f
g
monton wachsend.
□ Angenommen, f(x), g(x) ̸= 0 für alle x ∈ D. Dann ist f
q
oder g
f monoton
wachsend.
(b) Kreuze die richtigen Aussagen an. Sei f : R → R eine Funktion, die stetige bei
x0 = 0 ist mit f(x0) > 0.
□ Es existieren ε, δ > 0 so dass f(x) > ε für alle x ∈ (−δ, δ) gilt.
□ Es gilt f(x) ≥ 0 für alle x ∈ R.
□ Beide obige Aussagen sind falsch.
(c) Kreuze die richtigen Aussagen an.
□ f : [0, 1] → R beschränkt =⇒ f monoton.
□ f : [0, 1] → R strikt monoton wachsend =⇒ f stetig.
□ f : (0, 1] → R monoton =⇒ f beschränkt.
□ f : [0, 1] → R monoton =⇒ f beschränkt.
(d) Die Aufrundungsfunktion R → R, x 7→ ⌈x⌉ := min{n ∈ Z | n ≥ x} ist im Punkt
x = 2
□ stetig.
□ unstetig.
□ Die Informationen genügen nicht um zu schliessen.
1/2
ETH Zürich
FS 2022
Analysis I
Serie 6
d-infk
Prof. Dr. Özlem Imamoglu
*6.2. Cauchy Produkt. Zeigen Sie, dass für jedes x ∈ R wobei |x| < 1 Folgendes
gilt:
X∞
n=0
(n + 1)x
n =
1
(1 − x)
2
.
*6.3. Stetigkeit I. Zeigen Sie direkt aus der ε-δ Definition, dass
f : R → R, f(x) = e
x
in jedem Punkt stetig ist.
6.4. Stetigkeit II. Zeigen Sie, dass die Funktion
f : R → R, f(x) =



x, x ∈ Q,
0, x ∈ R \ Q
nur in x = 0 stetig ist.
2/2
