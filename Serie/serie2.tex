d-infk
Prof. Dr. Özlem Imamoglu
Analysis I
Serie 2
ETH Zürich
FS 2022
Nur die Aufgaben mit einem * werden korrigiert.
2.1. MC Fragen: Folgenkonvergenz. Wählen Sie die richtigen Antworten.
(a) Welche der Aussagen ist richtig?
 Eine Folge kann höchstens ein Grenzwert haben.
 Jede monotone und von oben beschränkte Folge ist konvergent.
 Es gibt konvergente Folgen, die nicht beschränkt sind.
 Eine divergente Folge ist nicht beschränkt.
(b) Seien (an), (bn) und (cn) Folgen in R mit cn = an + bn.
 Falls limn→∞ cn existiert, existieren limn→∞ an und limn→∞ bn, und es gilt:
limn→∞
an + limn→∞
bn = limn→∞
cn.
 Falls limn→∞ cn und limn→∞ bn existieren, existiert limn→∞ an und es gilt:
limn→∞
an = limn→∞
cn − limn→∞
bn.
 Falls (an) und (bn) beschränkt sind, muss (cn) beschränkt sein.
 Falls (cn) konvergiert, konvergiert mindestens eine der Folgen (an) und
(bn).
(c) Sei (an) eine Folge in R.
 Falls ε > 0 und a ∈ R existieren, so dass
|an − a| < ε ∀n ≥ 1
gilt, dann konvergiert (an).
 Falls (an) konvergiert, ist die Folge bn = an+1 + an konvergent.
 Falls die Folge bn = an+1 − an nach 0 konvergiert, ist (an) konvergent.
 Falls a ∈ R existiert, so dass an ≤ a ∀n ≥ 1, und an+1 ≥ an ∀n ≥ 1, dann
ist (an) konvergent.
2.2. Äquvalente Definitionen der Konvergenz. Sei (an) eine reelle Folge und
sei L ∈ R. Zeigen Sie, dass die folgenden Aussagen äquivalent sind:
1/2
ETH Zürich
FS 2022
Analysis I
Serie 2
d-infk
Prof. Dr. Özlem Imamoglu
(i) Für alle ε ist die Menge {n ∈ N | an 6∈ (L − ε, L + ε)} endlich;
(ii) Für alle ε existiert Nε ≥ 1, so dass |an − L| < ε für alle n ≥ Nε gilt.
*2.3. Grenzwert I. Sei a ∈ R, a > 0. Beweisen Sie, dass die folgende Gleichung
gilt:
lim n→+∞
√n
a = 1.
Hinweis: Der Binomialsatz könnte nützlich sein.
2.4. Grenzwert II. Man untersuche die nachstehenden Zahlenfolgen. Sind sie
beschra¨nkt? Konvergieren sie? Wenn ja: Welches ist ihr Grenzwert?
*(a) an =
3n
5 + 2n
3 + 5n
10 + 2n5
;
*(b) bn =
√
n2 + 3n − n;
(c) cn =
3
n + (−2)n
3
n − 2
n
;
(d) dn =

n
n2
+
n + 1
n2
+ · · · +
3n
n2

;
*(e) en =
√n
5
n + 11n + 17n.
2.5. Divergente Folgen. Finden Sie Beispiele für reelle Folgen (xn) und (yn), so
dass xn → +∞, yn → −∞ und
(a) xn + yn → +∞;
(b) xn + yn → −∞;
(c) (xn + yn) konvergiert;
(d) (xn + yn) beschränkt ist und divergiert.
2/2
