d-infk
Prof. Dr. Özlem Imamoglu
Analysis I
Serie 11
ETH Zürich
FS 2022
Nur die Aufgaben mit einem * werden korrigiert.
11.1. MC Fragen.
(a) Sei f : R → R dreimal stetig differenzierbar mit Graph:
x
f
′
(x)
Welche der folgenden Aussagen trifft zu?
□ f ist positiv,
□ f ist nicht monoton,
□ f besitzt eine Nullstelle,
□ f
′ besitzt eine Nullstelle,
□ f
′′ besitzt keine Nullstelle,
(b) Welche der folgenden Implikationsketten für eine Funktion f sind richtig?
□ f ist differenzierbar =⇒ f ist stetig.
□ f ist stetig =⇒ f ist differenzierbar.
□ f
′′ > 0 =⇒ f ist konvex.
□ f
′′ > 0 =⇒ f ist konkav.
(c) Wählen Sie die richtige Aussagen.
□ Falls fn stetige Funktionen sind und falls fn nach f gleichmässig konvergiert,
dann ist f auch stetig.
1/2
ETH Zürich
FS 2022
Analysis I
Serie 11
d-infk
Prof. Dr. Özlem Imamoglu
□ Falls fn differenzierbare Funktionen sind und falls fn nach f gleichmässig
konvergiert, dann ist f auch differenzierbar.
□ Falls fn eine Funktionenfolge ist, wobei fn einmal stetig differenzierbar ist
für jede n ∈ N und falls sowohl (fn) als auch (f
′
n
) gleichmässig konvergieren
mit fn → f und f
′
n → g, dann ist auch f stetig differenzierbar mit f
′ = g.
*11.2. n-te Ableitung. Berechnen Sie die n-te Ableitung von die folgende Funktionen:
(a) 1
1 − 5x + 6x
2
(b) sin(6x) cos(4x)
*11.3. Nullstellen von Funktionen.
(a) Zeige, dass die Funktion f : R → R, f(x) = arctan(x) nur für x = 0 verschwindet.
(b) Zeige, dass die Funktion f : R → R, f(x) = x
3 − 6x
2 − 1 nur in einer Punkt
verschwindet.
11.4. Extrema. Finden Sie die Extrema von
f : R → R, f(x) = √3
x
2 − x
3
.
2/2
