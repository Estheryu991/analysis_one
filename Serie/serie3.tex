d-infk
Prof. Dr. Özlem Imamoglu
Analysis I
Serie 3
ETH Zürich
FS 2022
Nur die Aufgaben mit einem * werden korrigiert.
3.1. MC Fragen: Folgenkonvergenz. Wählen Sie die richtigen Antworten.
(a) Sei an definiert durch
an =



1 + q k
12k+1 n = 3k + 1 für k ≥ 0,
5k
3+k
k
3+1 n = 3k + 2 für k ≥ 0,
(−1)k
k
n = 3k + 3 für k ≥ 0.
Welche der Aussagen gilt?
□ limn→∞ an existiert.
□ lim infn→∞ an existiert.
□ lim supn→∞ an = 1 + q
1/12
(b) Welche der folgenden Aussagen sind richtig?
□ Sei (qn)n≥1 eine Folge rationaler Zahlen, sodass
|qn − qn+1| → 0 für n → ∞.
Dann ist (qn)n∈N eine Cauchy-Folge.
□ Sei (an)n≥1 eine konvergente Folge, und σ eine Permutation von {1, 2, 3, . . . }
(d.h. eine Bijektion der Menge {1, 2, 3, . . . } auf sich selbst). Dann konvergiert auch die Folge (bn)n≥1, bn = aσ(n)
, ∀n ≥ 1.
(c) Sei (xn)n eine Cauchy-Folge in R. Dann
□ konvergiert P∞
n=1 √
xn;
□ kann (xn)n unbeschränkt sein;
□ gibt zu jedem ε > 0 ein N ∈ N so dass für alle m, n > N
|xm − xn| < ε.
1/2
ETH Zürich
FS 2022
Analysis I
Serie 3
d-infk
Prof. Dr. Özlem Imamoglu
*3.2. Grenzwert. Bestimmen Sie die folgenden Grenzwerte:
(a) limn→∞
√n
3n + 4 ;
(b) limn→∞
(−19)n−π
13n+1 ;
(c) limn→∞
(−2022)n+(−2023)n
(−2022)n+1−(−2023)n ;
(d) limn→∞

1 + 1
n3
n
2
.
*3.3. Fibonacci. Die reelle Folge (an)n≥1 sei rekursiv gegeben durch
a1 = 1, a2 = 1, an+1 = an + an−1 für n ≥ 2.
(a) Beweisen Sie folgende explizite Formel durch vollständige Induktion.
an =
1
√
5
 
1 + √
5
2
n
−

1 −
√
5
2
n
!
.
(b) Zeigen Sie, dass bn :=
an+1
an
gegen die goldene Zahl Φ :=
1+√
5
2
konvergiert.
(c) Finden Sie eine Zahl n ≥ 1 sodass folgende Aussage gilt:
∀m ∈ N, m ≥ n :


bm −
1+√
5
2


 ≤
1
100 .
3.4. Bernoulli Ungleichung. Zeigen Sie, dass für jedes n ∈ N und x ∈ R, wobei
x > −1, Folgendes gilt:
(1 + x)
n ≥ 1 + nx.
2/2
