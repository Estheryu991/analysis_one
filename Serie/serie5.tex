d-infk
Prof. Dr. Özlem Imamoglu
Analysis I
Lösung von Serie 5
ETH Zürich
FS 2022
Nur die Aufgaben mit einem * werden korrigiert.
5.1. MC Fragen: Folgenkonvergenz Wählen Sie die richtigen Antworten.
(a) Wir nehmen an, dass P∞
n=1 cn absolut konvergiert und α > 0. Definiere:
an = cnα
n
bn = ncnα
n−1
Welche Aussage trifft zu?
□ lim sup
n→+∞
|an|
1/n > lim sup
n→+∞
|bn|
1/n
.
□ lim sup
n→+∞
|an|
1/n < lim sup
n→+∞
|bn|
1/n
.
□ lim sup
n→+∞
|an|
1/n = lim sup
n→+∞
|bn|
1/n
.
□ Die Informationen genügen nicht um zu schliessen.
(b) Wir nehmen an, dass X∞
k=1
ak absolut konvergiert und dass X∞
k=1
bk konvergiert.
Geben Sie die korrekte Antwort auf folgende zwei Fragen an.
(A) Die Reihe X∞
k=1
|ak|
2
□ konvergiert nicht unbedingt.
□ konvergiert immer, aber konvergiert nicht unbedingt absolut.
□ konvergiert immer absolut.
□ keine der obigen Aussagen trifft zu.
(B) Die Reihe X∞
k=1
akbk
□ konvergiert nicht unbedingt.
□ konvergiert immer, aber konvergiert nicht unbedingt absolut.
□ konvergiert immer absolut.
□ keine der obigen Aussagen trifft zu.
Zuletzt geändert: 27. März 2022 1/2
ETH Zürich
FS 2022
Analysis I
Serie 5
d-infk
Prof. Dr. Özlem Imamoglu
(c) Wir nehmen an, dass X∞
k=1
ak divergiert und dass X∞
k=1
bk divergiert. Die Reihe
X∞
k=1
akbk
□ konvergiert nicht unbedingt.
□ konvergiert immer, aber konvergiert nicht unbedingt absolut.
□ konvergiert immer absolut.
□ keine der obigen Aussagen trifft zu.
5.2. Wurzelkriterium „starker” als Quotientkriterium. Zeigen Sie, dass
Folgendes gilt:
lim inf




an+1
an




≤ lim inf |an|
1/n ≤ lim sup |an|
1/n ≤ lim sup




an+1
an




.
*5.3. Reihen I Untersuchen Sie folgende Reihen auf Konvergenz und absolute
Konvergenz.
(a) X∞
n=1
√
n + 2 −
√
n + 1
n
(b) X∞
n=1
n!(2n)! sin(n
17)
(3n)!
(c) X∞
n=1
5n + 2n
3
n
*5.4. Reihen II Finden Sie den Konvergenzradius von
X∞
i=1
(n!)3
(3n)!x
n
.
2/2
