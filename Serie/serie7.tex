d-infk
Prof. Dr. Özlem Imamoglu
Analysis I
Serie 7
ETH Zürich
FS 2022
Nur die Aufgaben mit einem * werden korrigiert.
7.1. MC Fragen.
(a) Wählen Sie alle Funktionen, die in jedem Punkt ihres Definitionsbereichs stetig
sind.
□ f : R → R, f(x) = (4x − 6)12 + x
4
x
2 + 1
;
□ f : R → R, f(x) = 1
x
;
□ f : R → R, f(x) = |x|;
□ f : R → R, f(x) = sign(x), wobei
sign(x) =



+1, x > 0,
0, x = 0,
−1, x < 0.
□ f : R → R, f(x) = |x| · sign(x).
(b) Sei I ein Intervall und f : I → R eine stetige Funktion. Kreuzen Sie die richtigen
Aussagen an.
□ Falls I kompakt ist, ist auch f(I) kompakt.
□ Falls I kompakt ist, ist f(I) nicht unbedingt kompakt.
□ Falls f(I) kompakt ist, ist auch I kompakt.
(c) Welche der folgenden Aussagen ist korrekt? In allen Fällen seien a, b reelle
Zahlen mit a < b.
□ Sei f : [a, b] → R eine stetige Funktion und es gelte f(a) < f(b). Dann
liegen alle Funktionswerte zwischen f(a) und f(b).
□ Sei f : [a, b] → R eine monoton wachsende stetige Funktion mit f(a) ≤
0 ≤ f(b). Dann besitzt f in [a, b] genau eine Nullstelle.
□ Sei f : [a, b] → R eine streng monoton wachsende stetige Funktion mit
f(a) < 0 < f(b). Dann besitzt f in (a, b) genau eine Nullstelle.
*7.2. Umkehrfunktion. Analysiere folgende Funktionen auf strikte Monotonie, und
falls möglich, bestimme die Inverse Funktion.
1/2
ETH Zürich
FS 2022
Analysis I
Serie 7
d-infk
Prof. Dr. Özlem Imamoglu
(a) f(x) = 4 · ln(x + 7) + 3 für x ∈ (−7, +∞),
(b) f(x) = e
x − e
−x
2
für x ∈ R,
(c) f(x) = e
−x
2
für x ∈ R.
7.3. Zwischenwertsatz II. Beweisen Sie, dass am Äquator der Erde es immer zwei
gegenüberliegende Punkte mit gleicher Temperatur gibt.
Hinweis: Nehmen Sie an, dass die Temperatur durch eine stetige Funktion dargestellt
werden kann, und betrachten Sie die Temperaturdifferenz zwischen Antipodenpunkten
auf einem Grosskreis.
*7.4. Surjektivität von x
n
. Zeigen Sie, dass die Funktion
f : [0,∞) → [0,∞), x 7→ x
n
surjektiv ist.
2/2
