d-infk
Prof. Dr. Özlem Imamoglu
Analysis I
Serie 4
ETH Zürich
FS 2022
Nur die Aufgaben mit einem * werden korrigiert.
4.1. MC Fragen.
(a) Sei Xn =

0,
1
n
i
und Yn = [n, +∞) für n ≥ 1. Welche Aussagen sind richtig?
□ Xn ⊇ Xn+1 für jedes n ≥ 1;
□
T
n≥1 Xn ̸= ∅;
□ Yn ⊆ Yn+1 für jedes n ≥ 1;
□
T
n≥1 Yn = ∅.
(b) Sei P∞
n=1 an eine Reihe. Welche Aussagen sind richtig?
□
P∞
n=1 an konvergiert, falls limn→∞
an = 0.
□
P∞
n=1 an konvergiert, falls die Folge (Sm) der Partialsummen Sm =
Pm
n=1 an
konvergiert.
□ Falls P∞
n=1 bn eine konvergente Reihe ist, wobei 0 ≤ bn ≤ an für jedes
n ∈ N, dann konvergiert P∞
n=1 an.
(c) Sei ϕ : N
∗ → N
∗
eine Abbildung, P∞
n=1 an eine Reihe und bn = aϕ(n)
. Welche der
folgenden Aussagen stimmt?
□
P∞
n=1 an ist konvergent und ϕ surjektiv =⇒
P∞
n=1 bn ist konvergent.
□
P∞
n=1 an ist konvergent und ϕ injektiv =⇒
P∞
n=1 bn ist konvergent.
□
P∞
n=1 an ist absolut konvergent und ϕ surjektiv =⇒
P∞
n=1 bn ist konvergent.
□
P∞
n=1 an ist absolut konvergent und ϕ injektiv =⇒
P∞
n=1 bn ist konvergent.
*4.2. Limit-Vergleichssatz Sei (an) und (bn) zwei Folgen, wobei an, bn > 0 für
jedes n ∈ N. Angenommen, es existiert eine reelle Zahl l > 0, so dass limn→∞
an/bn = l.
(a) Zeigen Sie, dass es ein N > 0, wobei l
2
bn < an <
3l
2
bn für jedes n > N, gibt.
(b) Zeigen Sie, dass P∞
n=1 an genau dann konvergiert, wenn P∞
n=1 bn konvergiert.
(c) Das Obenstehende gilt nicht, wenn an/bn → 0. Finden Sie ein Beispiel, bei dem
an, bn > 0, an/bn → 0 und P∞
n=1 an konvergiert, aber P∞
n=1 bn divergiert.
*4.3. Reihe I. Untersuche das Konvergenzverhalten folgender Reihen. Wenn die
Reihe konvergiert, berechnen Sie den Wert der Reihe.
1/2
ETH Zürich
FS 2022
Analysis I
Serie 4
d-infk
Prof. Dr. Özlem Imamoglu
(a) X∞
n=3
1
n(n − 1)(n + 1),
(b) X∞
n=2
6 · 2
n−2
3
n
,
(c) X∞
n=1
3
2n + 2
.
4.4. Reihe II. Sei P∞
k=1 an konvergent, mit ak ≥ 0, ∀k ≥ 1. Beweisen Sie, dass
X∞
k=1
√
akak+1
konvergiert.
2/2
