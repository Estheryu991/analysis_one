d-infk
Prof. Dr. Özlem Imamoglu
Analysis I
Serie 13
ETH Zürich
FS 2022
Nur die Aufgaben mit einem * werden korrigiert.
13.1. MC Fragen Wählen Sie die richtigen Antworten.
(a) Für f ∈ C
0
(R) und g ∈ C
1
(R) mit −∞ < a < b < +∞ lautet die Substitutionsregel

Z g(b)
g(a)
f(g(x))g
0
(x) dx =
Z b
a
f(t) dt

Z b
a
f(g(x))g
0
(x) dx =
Z g(b)
g(a)
f(t) dt

Z b
a
f(
x
2
2
)x dx =
Z b
2
2
a2
2
f(t) dt

Z b
a
f(
x
2
2
) dx =
Z b
2
a
2
tf(t) dt
(b) Die Ableitung nach x von g(x) = Z 1
x2
sin2
(t) cos2
(t)dt ist
 g
0
(x) = Z 0
2x
sin2
(t) cos2
(t)dt.
 g
0
(x) = − sin2
(x
2
) cos2
(x
2
).
 g
0
(x) = −2x sin2
(x
2
) cos2
(x
2
).
*13.2. Flächeninhalt einer Form. Berechnen Sie die Fläche, die durch die lineare
Funktion (blau) und die quadratische Funktion (rot) begrenzt ist, wie im Bild unten
gezeigt.
1/2
ETH Zürich
FS 2022
Analysis I
Serie 13
d-infk
Prof. Dr. Özlem Imamoglu
13.3. Integration I. Für zwei ganze Zahlen p, q ≥ 0 definieren wir
I(p, q) := Z 1
0
x
p
(1 − x)
q
dx .
Zeigen Sie, dass
I(p, q) = p! q!
(p + q + 1)!
Hinweis: Bestimmen Sie mit Hilfe einer partiellen Integration eine Rekursionsrelation
zwischen den Grössen I(p + 1, q) und I(p, q + 1) und berechnen Sie I(p, 0).
13.4. Integration II. Berechnen Sie folgende bestimmte oder unbestimmte Integrale:
*(a) Z 7
1
4 − x
3 + x
x
dx; (b) Z 2
1
(x
2/3 − 2) (x
2 + 3) dx;
*(c) Z
cos(cos x) sin x dx; *(d) Z 1
0
t
2
cos(2t) dt;
(e) Z π/4
0
1 − cos2 x
2 cos2 x
dx; (f) Z
(x
4 + 4x + 4)2022(4x
3 + 4) dx;
*(g) Z
e
6x
· sin(3x) dx; (h) Z
2x
√
3 + 4x
2
dx.
2/2
