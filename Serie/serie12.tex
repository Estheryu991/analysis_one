d-infk
Prof. Dr. Özlem Imamoglu
Analysis I
Serie 12
ETH Zürich
FS 2022
Nur die Aufgaben mit einem * werden korrigiert.
12.1. MC Fragen.
(a) Der Wert des Integrals R 1
−1
|x| dx beträgt
□ 0
□
1
2
□ 1
□ 2
(b) Sei f : [a, b] → R eine Funktion. Wählen Sie die richtige Aussagen:
□ f ist immer integrierbar.
□ Falls f monoton ist, ist f auch integreirbar.
□ Falls f beschränkt ist, ist f auch integrierbar.
□ Falls f stetig ist, ist f auch integrierbar.
(c) Ist die folgende Aussage wahr?
Sei f : [a, b] → R eine Funktion. Dann gibt es c ∈ [a, b] mit
Z b
a
f(x) dx = f(c)(b − a).
□ Ja
□ Nein
12.2. Integration I.
(a) Überprüfen Sie nach der Definition, ob das Folgende gilt:
Z b
a
c dx = c · (b − a),
wobei a, b, c ∈ R.
(b) Berechnen Sie
Z 4
0
⌊x⌋dx.
1/2
ETH Zürich
FS 2022
Analysis I
Serie 12
d-infk
Prof. Dr. Özlem Imamoglu
*12.3. Integration II. Berechnen Sie die Integral
Z 2
1
1
x
2
dx
nach der Definition.
Hinweis: Als Partition nehmen Sie P =
n
xk
:= 1 + k
n
| k ∈ {1, 2, . . . , n}
o
und als
Mittelpunkte nehmen Sie ξi =
√xi
· xi+1.
*12.4. Stammfunktionen
(a) Seien f : [a, b] → R und g : [c, d] → R differenzierbare Funktionen mit [c, d] ⊆
f([a, b]). Bestimmen Sie eine Stammfunktion zu
x 7→ f
′
(g(x))g
′
(x), x ∈ [c, d].
Finden Sie eine Stammfunktion der folgenden Funktionen:
(b) (x
2 − 2x + 2)2022(2x − 2);
(c) − e
1/x 1
x
2
;
(d) x
√
1 + x
2
;
(e) f
′
(x)
f(x)
, mit f beliebig ;
(f) tan x.
2/2
