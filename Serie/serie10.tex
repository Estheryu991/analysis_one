d-infk
Prof. Dr. Özlem Imamoglu
Analysis I
Serie 10
ETH Zürich
FS 2022
Nur die Aufgaben mit einem * werden korrigiert.
10.1. MC Fragen.
(a) Kreuze die richtigen Aussagen an
□ Ist der Graph einer Funktion eine Gerade, dann ist die zugehörige Ableitung
konstant.
□ Ist eine Funktion f das Doppelte einer Funktion g, dann ist auch die
Ableitung von f das Doppelte der Ableitung von g.
□ Ist f(0) < 0, dann gilt auch f
′
(0) < 0.
(b) Seien a < b relle Zahlen, g : R → R eine beschränkte Funktion und
f : [a, b] → R eine beschränkte Funktion mit f(a) < f(b). Welche Aussagen
treffen zu?
□ Falls es für jedes c ∈ [f(a), f(b)], ein x ∈ [a, b] gibt mit f(x) = c, so folgt,
dass f stetig ist.
□ Falls g ◦ f und g differenzierbar sind, so folgt, dass f differenzierbar ist.
□ Falls f differenzierbar ist, gibt es x0 ∈ [a, b] so, dass
f
′
(x0) = f(b) − f(a)
b − a
.
(c) Sei f : R → R und h: R → [0, +∞) so dass
lim x→+∞
f(x) = −∞, limx→∞
h(x) = 0.
Kreuzen Sie die Richtige Aussagen an.
□ limx→∞ h(x) · f(x) = 0 ist möglich.
□ limx→∞ h(x) · f(x) = +∞ ist möglich.
□ limx→∞ h(x) · f(x) = −∞ ist nicht möglich.
□ limx→−∞ h(x) · f(x) = +∞ ist möglich.
*10.2. Ableitung I.
Zeige:
1/2
ETH Zürich
FS 2022
Analysis I
Serie 10
d-infk
Prof. Dr. Özlem Imamoglu
(a)
tan: 
−
π
2
,
π
2

→ R
ist streng monton wachsend.
(b)
lim
x→(
π
2 )
−
tan(x) = +∞ und lim
x→(− π
2 )
+
tan(x) = −∞.
(c) Schliesse, dass tan: 
−
π
2
,
π
2

→ R bijektiv ist. Die inverse wird arctan gennant.
(d) Berechne die Ableitung von arctan.
10.3. Ableitung II. Berechnen Sie die Ableitung der f : R \ {0} → R, f(x) =
x sin 
1
x

.
Ist
g : R → R, g(x) =



x sin 
1
x

, x ̸= 0
0, x = 0
in x = 0 differenzierbar?
*10.4. L’Hospital Regel. Berechnen Sie die Folgende Grenzwerte.
(a) limx→3
x
3 − 4x
2 + 9
x
2 + x − 12
(b) limx→∞
ln(2x)
x
2
(c) limx→∞
(e
x + x)
1/x
2/2
