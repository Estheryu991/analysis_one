d-infk
Prof. Dr. Özlem Imamoglu
Analysis I
Serie 1
ETH Zürich
FS 2022
Nur die Aufgaben mit einem * werden korrigiert.
1.1. MC Fragen: Supremum und Infimum auf R. Wählen Sie die einzig richtige
Antwort.
(a) 2 ist eine obere Schranke von [0, 1).
 Ja
 Nein
(b) Wenn A ⊂ B und A ein Maximum besitzt, dann besitzt auch B ein Maximum.
 Ja
 Nein
(c) min{
k
k+2 | k ∈ N} = 0. Hier ist N = {0, 1, 2, . . .} die Menge der natürlichen
Zahlen.
 Ja
 Nein
(d) Sei S eine nichtleere, nach oben beschränkte Teilmenge von R und sei a ∈ R ihr
Supremum. Dann gilt:
 für jedes ε > 0 existiert eine obere Schranke b von S, so dass a − ε < b < a;
 S \ {a} besitzt ein Maximum;
 a ist das Infimum der oberen Schranken.
*1.2. Axiome der reellen Zahlen. Zeigen Sie, dass für alle x, y, u, v ∈ R, wobei
x ≤ y und u ≤ v, folgendes gilt:
x + u ≤ y + v.
1.3. Supremum und Infimum I. Seien a, b ∈ R, mit a > 0 und S eine nichtleere,
von oben beschränkte Menge. Beweisen Sie, dass folgendes gilt:
sup
x∈S
(ax + b) = a sup
x∈S
x + b.
1/2
ETH Zürich
FS 2022
Analysis I
Serie 1
d-infk
Prof. Dr. Özlem Imamoglu
*1.4. Supremum und Infimum II. Bestimmen Sie, falls vorhanden, das Infimum,
Supremum, Minimum und Maximum der folgenden Teilmengen der reellen Zahlen:
A1 =
n
x
2 − 5x + 6 | x ∈ R
o
,
A2 =

1
2 + k
+
1
3 + m


 k, m ∈ N

.
1.5. Komplexe Zahlen - Wiederholung. Finden Sie für jede der folgenden
komplexen Zahlen z
• ihre kartesische Form A + iB,
• ihren Betrag |z|,
• ihr Konjugiertes z¯,
• ihr Reziprokes 1/z (in kartesischer Form):
z1 = −42, z2 = −
1
i
, z3 =
1 − i
1 + i
,
z4 = cos α + isin α, z5 = sin α + i cos α,
z6 = 2022 + i
2021, z7 = (1 + i)
6
,
wobei α ∈ R.
Hinweis: Vielleicht möchten Sie z7 zuerst in trigonometrischer Form schreiben.
Bemerkung: Die kartesische Form darf nicht i in dem Nenner erhalten! Z.B. 1 + i ist
OK, 1/(1 + i) nicht.
2/2
